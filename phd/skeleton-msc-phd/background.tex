%
% File: background.tex
%
% Date: ? 
%
% Description:
%   The background is given to ...
%
%
\chapter{Background to Graduate Research} \label{chap:background}

\summary{Give chapter summary}

\section{What Graduate Research is All About}
The distinguishing mark of graduate research is an original contribution to knowledge. The thesis is a formal document whose sole purpose is to prove that you have made an original contribution to knowledge. Failure to prove that you have made such a contribution generally leads to failure.

To this end, your thesis must show two important things:
\begin{itemize}
  \item you have identified a worthwhile problem or question which has not been previously answered,
  \item you have solved the problem or answered the question.
\end{itemize}
    
Your contribution to knowledge generally lies in your solution or answer. 

\section{What the Graduate Thesis is All About}
Because the purpose of the graduate thesis is to prove that you have made an original and useful contribution to knowledge, the examiners read your thesis to find the answers to the following questions:
\begin{itemize}
  \item what is this student's research question?
  \item is it a good question? (has it been answered before? is it a useful question to work on?)
  \item did the student convince me that the question was adequately answered?
  \item has the student made an adequate contribution to knowledge?
\end{itemize}
    
A very \emph{clear} statement of the question is essential to proving that you have made an original and worthwhile contribution to knowledge. To prove the originality and value of your contribution, you must present a thorough review of the existing literature on the subject, and on closely related subjects. Then, by making direct reference to your literature review, you must demonstrate that your question (a) has not been previously answered, and (b) is worth answering. Describing how you answered the question is usually easier to write about, since you have been intimately involved in the details over the course of your graduate work.

If your thesis does not provide adequate answers to the few questions listed above, you will likely be faced with a requirement for major revisions or you may fail your thesis defence outright. For this reason, the generic thesis skeleton given below is designed to highlight the answers to those questions with appropriate thesis organization and section titles. The generic thesis skeleton can be used for any thesis. While some professors may prefer a different organization, the essential elements in any thesis will be the same. Some further notes follow the skeleton.

Always remember that a thesis is a \emph{formal} document: every item must be in the appropriate place, and repetition of material in different places should be eliminated. 