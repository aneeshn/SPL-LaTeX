\documentclass[cover]{exam}
%\documentclass[cover,comments]{exam}\usepackage{draftcopy}\draftcopyName{ANSWER}{160}

\usepackage{ArcMacro,shadow,latexsym,epsf}
%\usepackage{amsmath}
\coursecode{ELEN224}
\coursename{Electronics I}
\exammonth{June 2000}
\studyyear{Second}
\studydegs{B.Sc (Eng) Elec.}
\internal{Dr. A. R. Clark \ \ x7223}
\external{Dr. G. J. Gibbon}
\duration{Three}
\instruct{Answer \emph{ALL} questions.\\Type `2' Examination.}

\begin{document}
\makeheads

\comment{\large \bfseries This version printed with the ``comments''
option\\\centering NOT FOR PRINTING!!!!!!!!!!} 

\begin{centre}
\shabox{Note: Show {\Large\em ALL\/} workings, complete with the
necessary comments!!---regardless of how fast your calculator can print the
results in one step. I am not interested in how well you can read your
formulae from your formula sheet. I am marking your reasoning, not only the
answer!! Marks are awarded for the reasoning as well as the ``answer''. A
correct numerical answer will not necessarily attract any marks!}
\end{centre}
\def\vs{\vspace{1ex}}                                                 

\question 
\comment{Op-Amp Question}
%KC 2.50
An inverting amplifier with a gain of -100V/V and an input resistance of
100k\Ohm, uses an op-amp with a 1mV offset voltage, a bias current of 30nA,
and an offset bias current of 3nA. 

What output voltage offset results with \\a) a basic uncompensated design
\\b) a bias-current compensated design (where a resistor is placed from the
non-inverting terminal to ground). 

Which offset source dominates in each case?

\comment{KC 2.50

This question will frighten most :-)

G=100, Rin = 100k\Ohm (=R1), hence R2=10M\Ohm 

In the Uncompensated case, both the effects of the offset voltage and the
bias currents cause a voltage at the inverting input. This voltage is then
multiplied by the closed loop gain to form the output offset. The bias
currents suck current from the efffective parallel combination of R1 and
R2.

Hence the output voltage becomes
\begin{align*}
v_0&=(1+R2/R1)(Vos+Vbias)
&=101(1mV + 30\times 10^9 \times 100k||10M
&=101(1mV + 3mV)
&=404mV
\end{align*}

Note that the voltage from the bias currents dominates the output offset.

For the bias-current compensated design, $R_3 = R_1//R_2$, and the voltage
at both terminals is raised due to the bias currents, and the output
voltage offset is then cancelled. The only remaining cause of output offset
is then the Vos and the Ios. In the worst case, 

\begin{align*}
v_0&=101(1mV + 3\times 10^9 \times 100k)
&=101(1mV + 0.3mV)
&=131mV
\end{align*}

Note that the Vos dominates the output.
}
\marks{20}




\question
\comment{Digital Question -- dead simple stuff.}
\Banum
\item 
Simplify the following to a minimum number of gates:
\begin{center}\input{99Test.cct}\ \box\graph\end{center}
\marks{10}
\comment{Trivial}
\item Design a state-machine which cycles between 5 possible states.
\marks{15}
\comment{ie a counter with simple combinational logic to reset it. They
have the choice of flip flops and polarity of edge triggering.}

\Eanum
\marks{Total 25}

\question
\comment{Diode Question}
Design a dual-ended power supply for an audio amplifier. It is required
that the rails are at $\pm$50V, and that each rail can deliver 4A. Justify
all assumptions. \emph{Hint: By ``design'', I mean specify the
characteristics and ratings of all components used.}
\marks{20}

\comment{
Simple stuff, but will get a wide range of allowable ripple etc.. I am
looking here for a bit of engineering judgement in sizing the caps. Can they
actually relate the 4A to a discharge rate and hence a voltage drop.

If they do include a voltage regulator, the lowest ripple voltage must
still be 2V above the output voltage etc.

Only twist is the dual-ended bit.

Clearly no ``answer'' in a numerical form.}



\question
\Banum
\item
Fully design and specify a circuit to turn on a security floodlight at
night.  Amongst other possible components, use an LDR (Light Dependant
Resistor) and an NPN Transistor. Make reasonable assumptions about the LDR
characteristics.

\comment{I am looking for a simple but thorough design where the LDR is
fed by a resistor. The question is really a check on whether they
individually designed their project, where the LDR was used with op-amps to
do the same. 

As light decreases, the voltage at the LDR increases, and will turn on the
NPN transistor. The subtlety comes in the fact that the base drive needs
supplying from the potential divider too. 

Hence sizing of the resistors is important. Most of the ``feel'' for the
LDR etc will have come from the project. They must recognise that a relay
will be needed for the mains. Bonus if they derive a non transformered
power source!!!!

Bonus plus if they actually use a transistor based Schmidt Trigger

(The relay itself will provide sufficient hysteresis, but I don't expect
anyone to actually mention this!)}

\marks{15}

\item With the aid of sketches, explain how an N-channel enhancement
MOSFET first gains its channel. Further explain what happens when the
transistor conducts an appreciable current. \marks{10}
\comment{A little bit of bookwork. Need to see gate-body rise to give
depletion region, followed by n-channel forming after the threshold
voltage. As current increases, channel skews and pinches off. Alleviated by
incresing gate voltage.}
\Eanum
\marks{Total 25}




\question
\comment{Transistor Question (SSEx6.7)}
  Consider the following circuit:
    \begin{center}\input{Mirror.cct}\ \box\graph\end{center}
\Banum\item
  For the current mirror shown above using two matched-gain transistors,
  find the value of $R$ that results in $I_o=1$mA with $V_{CC}$=5V\\
    i)Assume $\beta=\infty$\\
    ii)Assume $\beta=100$\\
    iii)For case (ii), for what values of $V_o$ will the current mirror work?

\item In what application would the above circuit be used?.
\comment{Transistor biasing, long tailed pairs etc}
    
    \Eanum\marks {20}
\comment{Obviously, for infinite $\beta$, the current feeding into both
bases is negected. Hence R=(5-0.7)/1mA=4.3k

For $\beta=100$, however, a full analysis needs doing. Assuming that the
base drives are the same (matched). We now get $2i_B$ through $R$ as well. 
Hence $I_o=1mA, i_B=10\mu A, I_{ref}=1.02mA, R=(5-0.7)/1.02mA = 4.22k$ 
}

\marks{Total 20}
\marks{Exam Total 110}
\marks{100\%=100}




\filedescribe
\end{document}



























\marks{Exam Total 110}

\filedescribe
\end{document}
